\documentclass{article}
\usepackage[utf8]{inputenc}
\usepackage[T2A]{fontenc}
\usepackage[russian]{babel}
\usepackage{indentfirst}

\begin{document}

\section*{Вариант10}
\section*{\centerline{Лекция} \\ \centerline{\textit{Аппаратно-независимый уровень управления виртуальной}} \\ \centerline{\textit{памятью}}}

Большинство ОС используют сегментно-страничную виртуальную
память. Для обеспечения нужной производительности менеджер памяти ОС
старается поддерживать в оперативной памяти актуальную информацию,
пытаясь угадать, к каким логическим адресам последует обращение в
недалеком будущем.

\begin{enumerate}
    \item[\textbf{\textsl{1.}}] \textbf{\textit{Исключительные ситуации при работе с памятью}}
    \item[\textbf{\textsl{2.}}] \textbf{\textit{Стратегии управления страничной памятью}}
    \item[\textbf{\textsl{3.}}] \textbf{\textit{Алгоритмы замещения страниц}}
    \begin{enumerate}
        \item[\textbf{3.1.}] \textbf{Алгоритм FIFO. Выталкивание первой пришедшей страницы}
        \begin{enumerate}
            \item[\textbf{3.1.1.}] \textbf{Аномалия Билэди (Belady)}
        \end{enumerate}
        \item[\textbf{3.2.}] \textbf{Оптимальный алгоритм (ОРТ)}
        \item[\textbf{3.3.}] \textbf{Выталкивание дольше всего не использовавшейся страницы. Алгоритм LRU}
        \begin{enumerate}
            \item[\textsl{3.3.1.}] \textit{Выталкивание редко используемой страницы. Алгоритм NFU}
            \item[\textsl{3.3.2.}] \textit{Другие алгоритмы}
        \end{enumerate}
    \end{enumerate}
    \item[\textbf{\textsl{4.}}] \textbf{\textit{Управление количеством страниц, выделенных процессу. Модель рабочего множества}}
    \begin{enumerate}
        \item[\textbf{4.1}] \textbf{Трешинг (Thrashing)}
        \item[\textbf{4.2}] \textbf{Модель рабочего множества}
    \end{enumerate}
    \item[\textbf{\textsl{5.}}] \textbf{Страничные демоны}
    \item[\textbf{\textsl{6.}}] \textbf{Программная поддержка сегментной модели памяти процесса \newline{Отдельные аспекты функционирования менеджера памяти}}
\end{enumerate}


\end{document}