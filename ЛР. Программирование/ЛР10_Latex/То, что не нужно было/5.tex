\documentclass[14pt, a4paper]{extarticle} 
\usepackage[T2A]{fontenc}
\usepackage[utf8]{inputenc}
\usepackage[english, russian]{babel}
\usepackage{cmap} 
\usepackage{geometry} 
\geometry{left=3cm, right=1cm, top=2cm, bottom=2cm}
\usepackage{titlesec} 
\usepackage{tocloft} 
\usepackage{setspace} 
\onehalfspacing 
\usepackage{fancyhdr} 



\pagestyle{fancy}
\fancyhf{} 
\fancyhead[R]{\thepage} 
\renewcommand{\headrulewidth}{0pt}

\newcommand{\maketitlepage}{
    \thispagestyle{empty}
    \begin{center}
        \vspace*{-1.5cm}
        \textbf{МИНПРОСВЕЩЕНИЯ РОССИИ} \\
        \vspace{0.5cm}
        Федеральное государственное бюджетное образовательное учреждение \\
        высшего образования \\
        «Чувашский государственный университет им. И.Н.Ульянова » \\
        \vspace{2cm}
        \textbf{Реферат} \\
        по дисциплине: «Перспективные направления информатики и вычислительной техники» \\
        \vspace{2cm}
        {\fontsize{28}{34}\selectfont \textbf{Квантовые вычисления: принципы, современное состояние и перспективы}} \\
        \vspace{3cm}
        \begin{flushleft}
            \textbf{Выполнил:} студент гр. ФМ-11-25 \\
            Заметалин Д.А. \\
            \vspace{1cm}
            \textbf{Проверил:} \\
            д.т.н., проф. Дакому Нужныэти Проверки
        \end{flushleft}
        \vfill
        Чебоксары \\ 2025
    \end{center}
    \newpage
}

\titleformat{\section}[block]{\centering\bfseries\Large}{\thesection}{1em}{}
\titleformat{\subsection}[block]{\bfseries\large}{\thesubsection}{1em}{}
\titlespacing*{\section}{0pt}{\parskip}{-\parskip}
\titlespacing*{\subsection}{0pt}{\parskip}{-\parskip}

\renewcommand{\cfttoctitlefont}{\hfill\Large\bfseries}
\renewcommand{\cftaftertoctitle}{\hfill}
\renewcommand{\cftsecleader}{\cftdotfill{\cftdotsep}}
\setcounter{tocdepth}{2}


\newcommand{\startsection}[1]{
    \newpage
    \section{#1}
}

\begin{document}

\maketitlepage

\newpage
\tableofcontents
\thispagestyle{empty}

\startsection{Введение}

Актуальность темы квантовых вычислений в современном мире сложно переоценить. С момента появления классических компьютеров, основанных на архитектуре фон Неймана и принципах булевой алгебры, их мощность росла в соответствии с законом Мура, который предсказывал удвоение числа транзисторов на микросхеме примерно каждые два года. Однако в последнее десятилетие становится очевидным, что физические ограничения, такие как тепловыделение и квантовые эффекты на наноуровне, замедляют этот экспоненциальный рост. Это создает фундаментальный барьер на пути решения ряда сложнейших вычислительных задач, имеющих критическое значение для человечества.

К таким задачам относятся, например, точное моделирование сложных молекул для разработки новых лекарств и материалов, оптимизация глобальных логистических систем, взлом современных криптографических алгоритмов (таких как RSA) и создание мощных систем искусственного интеллекта. Классическим суперкомпьютерам для их решения потребовалось бы время, сравнимое с возрастом Вселенной. Квантовые вычисления предлагают принципиально иной подход, использующий законы квантовой механики для обработки информации, что потенциально позволяет решать эти задачи на порядки быстрее.

Целью данного реферата является анализ основ квантовых вычислений, оценка их текущего состояния и выявление ключевых перспектив развития. Для достижения этой цели в работе ставятся следующие задачи:
\begin{itemize}
    \item Изучить фундаментальные принципы квантовых вычислений (кубиты, суперпозиция, запутанность).
    \item Рассмотреть архитектуру квантового компьютера и основные технологические платформы для его реализации.
    \item Проанализировать наиболее известные квантовые алгоритмы и их потенциальные области применения.
    \item Оценить современный уровень развития квантовых технологий (NISQ-эра) и существующие вызовы.
    \item Обсудить перспективы и потенциальное влияние квантовых вычислений на различные сферы человеческой деятельности.
\end{itemize}

Практическая значимость работы заключается в систематизации знаний о быстро развивающейся области, что позволяет составить целостное представление о ее возможностях и ограничениях. Этот анализ может быть полезен студентам и специалистам в области информационных технологий, которые стремятся понять одно из самых перспективных направлений технологического прогресса.

\startsection{Основная часть}

\subsection{Фундаментальные принципы квантовых вычислений}

Основным элементом классического компьютера является бит, который может находиться строго в одном из двух состояний: 0 или 1. В отличие от него, фундаментальной единицей информации в квантовом компьютере является \textbf{кубит} (quantum bit). Кубит, благодаря принципу \textbf{суперпозиции}, может находиться не только в состояниях |0⟩ или |1⟩, но и в любой линейной комбинации этих состояний. Это математически описывается как $|\psi\rangle = \alpha|0\rangle + \beta|1\rangle$, где $\alpha$ и $\beta$ — комплексные числа, называемые амплитудами вероятностей, причем $|\alpha|^2 + |\beta|^2 = 1$. В момент измерения кубит коллапсирует в одно из базовых состояний (0 или 1) с вероятностью, определяемой квадратами модулей его амплитуд.

Вторым ключевым понятием является \textbf{квантовая запутанность}. Это особое состояние, в котором два или более кубитов оказываются взаимосвязаны таким образом, что состояние одного кубита невозможно описать независимо от состояния другого, даже если они физически разделены. Измерение одного запутанного кубита мгновенно влияет на состояние другого. Запутанность является ресурсом, который позволяет квантовым компьютерам обрабатывать информацию экспоненциально более эффективно по сравнению с классическими.

\subsection{Архитектура и технологические платформы квантовых компьютеров}

Архитектурно квантовый компьютер состоит из нескольких ключевых компонентов:
\begin{itemize}
    \item \textbf{Кубитовая процессорная часть}: массив кубитов, где производятся вычисления.
    \item \textbf{Система инициализации}: приведение кубитов в известное начальное состояние.
    \item \textbf{Система управления}: аппаратура для применения к кубитам последовательности квантовых гейтов (операций).
    \item \textbf{Система считывания}: высокоточные приборы для измерения конечного состояния кубитов.
    \item \textbf{Система охлаждения}: большинство платформ требуют экстремально низких температур, близких к абсолютному нулю, для поддержания когерентности кубитов.
\end{itemize}

На сегодняшний день существует несколько конкурирующих технологических платформ для реализации кубитов: сверхпроводящие контуры (компании Google, IBM), ионы в ловушках (IonQ, Honeywell), полупроводниковые квантовые точки (Intel) и топологические кубиты (Microsoft). Каждая из них имеет свои преимущества и недостатки в отношении времени когерентности, масштабируемости и точности операций.

\subsection{Ключевые квантовые алгоритмы и области применения}

Наиболее известными квантовыми алгоритмами, демонстрирующими превосходство над классическими, являются:
\begin{itemize}
    \item \textbf{Алгоритм Шора}: позволяет эффективно разлагать большие числа на простые множители, что ставит под угрозу современную асимметричную криптографию (RSA).
    \item \textbf{Алгоритм Гровера}: обеспечивает квадратичное ускорение при поиске в неструктурированной базе данных.
    \item \textbf{Квантовое моделирование (Фейнман)}: прямое моделирование квантовых систем (молекул, материалов), что недоступно для классических компьютеров.
\end{itemize}

Эти алгоритмы открывают путь к применению квантовых вычислений в фармацевтике (разработка новых лекарств), материаловедении (создание сверхпроводников), искусственном интеллекте (ускорение обучения моделей), финансовом моделировании и, конечно, в криптографии (как для взлома, так и для создания новых, квантово-стойких шифров).


\startsection{Заключение}

Проведенный анализ позволяет сделать вывод, что квантовые вычисления представляют собой не просто эволюцию, а настоящую революцию в области информационных технологий. Они основаны на принципиально ином способе обработки информации, использующем такие феномены квантовой механики, как суперпозиция и запутанность. Это дает им потенциал для экспоненциального ускорения решения определенного класса задач, которые являются практически нерешаемыми для самых мощных классических суперкомпьютеров.

В настоящее время отрасль находится в так называемой \textbf{NISQ-эре (Noisy Intermediate-Scale Quantum)} — эре зашумленных квантовых устройств среднего масштаба. Современные квантовые процессоры содержат от нескольких десятков до нескольких сотен кубитов, но они еще недостаточно стабильны и подвержены ошибкам из-за декогеренции и шумов. Основными вызовами на пути к созданию полноценного fault-tolerant (устойчивого к ошибкам) квантового компьютера являются:
\begin{itemize}
    \item Увеличение количества кубитов при сохранении их качества.
    \item Разработка эффективных методов коррекции квантовых ошибок.
    \item Создание специализированного программного обеспечения и алгоритмов, устойчивых к шуму.
    \item Развитие классической контролирующей инфраструктуры.
\end{itemize}

Несмотря на эти испытания, прогресс в области идет стремительными темпами. Крупные технологические компании, стартапы и государственные институты вкладывают значительные ресурсы в исследования и разработки. Уже сегодня квантовые компьютеры доступны в облаке, что позволяет исследователям и разработчикам по всему миру экспериментировать с ними.

В перспективе, по мере преодоления технологических барьеров, квантовые вычисления окажут трансформационное воздействие на науку, промышленность и безопасность. Они могут коренным образом изменить процессы разработки лекарств, открыть новые материалы с заданными свойствами, оптимизировать глобальные экономические системы и привести к пересмотру основ защиты информации. Таким образом, изучение и развитие квантовых технологий является не просто актуальной научной задачей, а стратегическим направлением, определяющим технологический суверенитет и конкурентоспособность государства в долгосрочной перспективе.

\startsection{Список использованных источников}

\begin{enumerate}
    \item Нильсен М., Чанг И. Квантовые вычисления и квантовая информация. — М.: Мир, 2016. — 824 с.
    \item Кайе Ф., Лафламм Р., Моска М. Введение в квантовые вычисления. — Ижевск: РХД, 2009. — 360 с.
    \item Preskill J. Quantum Computing in the NISQ era and beyond // Quantum. — 2018. — Vol. 2. — P. 79.
    \item Arute F. et al. Quantum supremacy using a programmable superconducting processor // Nature. — 2019. — Vol. 574. — P. 505–510.
    \item Официальный сайт компании IBM Quantum Experience. [Электронный ресурс]. — URL: https://quantum-computing.ibm.com (дата обращения: 01.11.2024).
    \item Официальный сайт компании Google AI Quantum. [Электронный ресурс]. — URL: https://ai.google/research/teams/applied-science/quantum-ai/ (дата обращения: 01.11.2024).
\end{enumerate}

\end{document}